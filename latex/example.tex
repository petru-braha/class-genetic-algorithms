\documentclass[10pt]{article}
\usepackage[utf8]{inputenc}
\usepackage[T1]{fontenc}
\usepackage{amsmath}
\usepackage{amsfonts}
\usepackage{amssymb}
\usepackage[version=4]{mhchem}
\usepackage{stmaryrd}
\usepackage{graphicx}
\usepackage[export]{adjustbox}
\graphicspath{ {./images/} }
\usepackage{hyperref}
\hypersetup{colorlinks=true, linkcolor=blue, filecolor=magenta, urlcolor=cyan,}
\urlstyle{same}

\title{Homework H1 }


\author{Bujență Lucian-Andrei}
\date{}


\begin{document}
\maketitle
Application of Hill Climbing and Simulated Annealing methods for optimizing mathematical functions.

November 8, 2023

\begin{abstract}
This research document investigates the problem of finding the minimum values for four different mathematical functions: De Jong 1, Schwefel's, Rastrigin's, and Michalewicz's. The study focuses on employing two optimization algorithms, Hill Climbing and Simulated Annealing, to tackle this problem. Three variants of Hill Climbing, namely the first improvement, best improvement, and worst improvement, are applied to search for the minimum values. Additionally, the Simulated Annealing algorithm is hybridized with one of the Hill Climbing variants to enhance its performance. The analysis is conducted on the $5-, 10$-, and 30 -dimensional versions of all functions. A precision of at least 5 decimal places after 0 is maintained throughout the experimentation. The research aims to evaluate the effectiveness and efficiency of these optimization techniques in minimizing the target functions and provide insights into their performance across different dimensions.
\end{abstract}

\section*{1 Introduction}
The optimization of mathematical functions plays a crucial role in various fields. In this research document, we tackle the problem of finding the minimum values for four different functions: De Jong 1, Schwefel's, Rastrigin's, and Michalewicz's. These functions pose challenges due to their complex landscapes and multiple local optima. To address this problem, we employ two popular optimization algorithms: Hill Climbing and Simulated Annealing. The Hill Climbing algorithm explores different improvement strategies, while Simulated Annealing utilizes a probabilistic approach. We investigate the performance of these algorithms across multiple dimensions, ranging from 5 to 30 . The objective is to analyze the effectiveness and efficiency of these algorithms in finding the global minimum for each function and provide insights into their performance in different dimensional spaces. The results of this research contribute to the understanding of optimization techniques and their applicability to solve complex optimization problems.

\section*{2 Method}
In the following I will present the functions along with their graphs and formulas.

\subsection*{2.1 De Jong 1 function.}
Function:

$$
f(x)=\sum_{i=1}^{n} x_{i}^{2}
$$

\begin{itemize}
  \item Search domain: $-5.12 \leq x_{i} \leq 5.12, \mathrm{i}=1,2, \ldots, \mathrm{n}$.
  \item Minimum for 5D, 10D and 30D: 0
  \item Function graph:\\
\includegraphics[max width=\textwidth, center]{2024_10_25_eadb717d73b8e0991fefg-2}
\end{itemize}

\subsection*{2.2 Rastrigin function.}
Function:

$$
f(x)=A \cdot n+\sum_{i=1}^{n}\left[x_{i}^{2}-A \cdot \cos \left(2 \pi x_{i}\right)\right], A=10, x_{i} \in[-5.12,5.15]
$$

\begin{itemize}
  \item Search domain: $-5.12 \leq x_{i} \leq 5.12, \mathrm{i}=1,2, \ldots, \mathrm{n}$.
  \item Minimum for 5D, 10D and $30 \mathrm{D}: 0$
  \item Function graph:\\
\includegraphics[max width=\textwidth, center]{2024_10_25_eadb717d73b8e0991fefg-3}
\end{itemize}

\subsection*{2.3 Michalewicz function.}
Function:

$$
f(x)=\sum_{i=1}^{n} x_{i}^{2}
$$

\begin{itemize}
  \item Search domain: $0 \leq x_{i} \leq \pi, \mathrm{i}=1,2, \ldots, \mathrm{n}$.
  \item Minimum for 5D: -4.68765
  \item Minimum for 10D: -9.66015
  \item Minimum for 30D: - 29.63057
  \item Function graph:\\
\includegraphics[max width=\textwidth, center]{2024_10_25_eadb717d73b8e0991fefg-3(1)}
\end{itemize}

\subsection*{2.4 Schewefel function.}
Function:

$$
f(x)=-\sum_{i=1}^{d} \sin \left(x_{i}\right) \sin ^{2 m}\left(\frac{i x_{i}^{2}}{\pi}\right), m=10
$$

\begin{itemize}
  \item Search domain: $-500 \leq x_{i} \leq 500, \mathrm{i}=1,2, \ldots, \mathrm{n}$.
  \item Minimum for 5D, 10D and 30D: 0
  \item Function graph:\\
\includegraphics[max width=\textwidth, center]{2024_10_25_eadb717d73b8e0991fefg-4}
\end{itemize}

\section*{3 Experiment}
To evaluate the performance of the Hill Climbing and Simulated Annealing algorithms in finding the minimum values for the De Jong 1, Schwefel's, Rastrigin's, and Michalewicz's functions, I've conducted a series of experiments. The experiments were conducted on functions with dimensions of 5,10 , and 30 . The precision of at least 5 decimal places after 0 was maintained throughout the experimentation.\\
For each function and dimension, the program executed the following steps:

\begin{enumerate}
  \item Initialization.\\
a) Randomly generated an initial solution within the search space boundaries.
  \item Hill Climbing.\\
a) Applied the best improvement variant of the Hill Climbing algorithm.\\
b) Iteratively selected a neighboring solution and evaluated it.\\
c) If the evaluation of the neighboring solution was better, it was accepted as the new current solution.\\
d) The process continued until no better solution was found, or a termination condition was met.\\
e) Repeated the Hill Climbing process with the first improvement and worst improvement variants.
  \item Simulated Annealing.\\
a) Hybridized the Simulated Annealing algorithm with the best improvement variant of Hill Climbing.\\
b) Defined an initial temperature and cooling schedule.\\
c) Iteratively generated a neighboring solution and evaluated it.\\
d) Compared the evaluation of the neighboring solution with the current solution.\\
e) Accepted the neighboring solution if it was better or based on a probability criterion influenced by the temperature and the difference in evaluation between the solutions.\\
f) Adjusted the temperature according to the cooling schedule until a termination condition was met.
  \item Performance Evaluation.\\
a) Recorded the best evaluated value obtained by each algorithm for each function and dimension.\\
b) Repeated the experiments multiple times to record all the results.
\end{enumerate}

This experiment was implemented using $\mathrm{C}++$ programming language.

\section*{4 Results}
HillClimbing - Best Improvement results:

\begin{center}
\begin{tabular}{|l|l|l|l|l|}
\hline
Function name & Michalewicz & Rastrigin & DeJong1 & Schewefel \\
\hline
Best solution found in 5D & -3.69887 & 0.00000 & 0.00000 & 0.00191 \\
\hline
Best solution found in 10D & -8.79125 & 2.19874 & 0.00000 & 27.74012 \\
\hline
Best solution found in 30D & -26.59712 & 22.89471 & 0.00000 & 1238.74912 \\
\hline
Worst solution found in 5D & -3.61231 & 1.98721 & 0.00000 & 1.70841 \\
\hline
Worst solution found in 10D & -8.21312 & 7.48367 & 0.00000 & 102.74982 \\
\hline
Worst solution found in 30D & -25.30192 & 31.76124 & 0.00000 & 1612.31241 \\
\hline
Avg.Time & 117.84016 s & 286.48471 s & 84.65817 s & 151.09861 s \\
\hline
\end{tabular}
\end{center}

HillClimbing - First Improvement results:

\begin{center}
\begin{tabular}{|l|l|l|l|l|}
\hline
Function name & Michalewicz & Rastrigin & DeJong1 & Schewefel \\
\hline
Best solution found in 5D & -3.69217 & 0.00000 & 0.00000 & 1.28491 \\
\hline
Best solution found in 10D & -8.46121 & 3.38132 & 0.00000 & 210.48912 \\
\hline
Best solution found in 30D & -25.98732 & 34.57912 & 0.00000 & 1452.24712 \\
\hline
Worst solution found in 5D & -3.62871 & 2.47812 & 0.00000 & 25.84901 \\
\hline
Worst solution found in 10D & -8.17851 & 9.89612 & 0.00000 & 291.34675 \\
\hline
Worst solution found in 30D & -24.89751 & 44.37092 & 0.00000 & 2119.44675 \\
\hline
Avg.Time & 25.41241 s & 34.75612 s & 19.86191 s & 49.65891 s \\
\hline
\end{tabular}
\end{center}

HillClimbing - Worst Improvement results:

\begin{center}
\begin{tabular}{|l|l|l|l|l|}
\hline
Function name & Michalewicz & Rastrigin & DeJong1 & Schewefel \\
\hline
Best solution found in 5D & -3.59182 & 0.00000 & 0.00000 & 55.74915 \\
\hline
Best solution found in 10D & -8.19411 & 3.19789 & 0.00000 & 224.71013 \\
\hline
Best solution found in 30D & -23.78012 & 33.89432 & 0.00000 & 1478.65893 \\
\hline
Worst solution found in 5D & -3.29015 & 2.13172 & 0.00000 & 81.74092 \\
\hline
Worst solution found in 10D & -7.89128 & 9.39804 & 0.00000 & 321.14392 \\
\hline
Worst solution found in 30D & -24.18910 & 44.08914 & 0.00000 & 2118.46891 \\
\hline
Avg.Time & 219.47191 s & 391.316112 s & 127.49981 s & 205.57191 s \\
\hline
\end{tabular}
\end{center}

Simulated Annealing results:

\begin{center}
\begin{tabular}{|l|l|l|l|l|}
\hline
Function name & Michalewicz & Rastrigin & DeJong1 & Schewefel \\
\hline
Best solution found in 5D & -4.48766 & 0.00000 & 0.00000 & 0.00145 \\
\hline
Best solution found in 10D & -9.45447 & 0.00000 & 0.00000 & 119.75987 \\
\hline
Best solution found in 30D & -28.36721 & 0.00000 & 0.00000 & 1641.75921 \\
\hline
Worst solution found in 5D & -4.52312 & 0.00000 & 0.00000 & 28.74062 \\
\hline
Worst solution found in 10D & -9.37861 & 4.99712 & 0.00000 & 422.47542 \\
\hline
Worst solution found in 30D & -27.68762 & 20.21312 & 0.00000 & 2149.25062 \\
\hline
Avg.Time & 120.84016 s & 109.74618 s & 92.12975 s & 131.89012 s \\
\hline
\end{tabular}
\end{center}

\section*{5 Conclusion}
In conclusion, this research document focused on the problem of finding the minimum values for four mathematical functions: De Jong 1, Schwefel's, Rastrigin's, and Michalewicz's. By utilizing the Hill Climbing and Simulated Annealing algorithms, we explored various optimization strategies to tackle these challenging functions. Through experimentation in different dimensional spaces (5, 10, 30), we evaluated the performance of these algorithms in terms of effectiveness and efficiency. The findings revealed that the choice of optimization algorithm and its variants significantly impacted the search for the global minimum. The Hill Climbing algorithm, with its first improvement, best improvement, and worst improvement variants, showcased different trade-offs between exploration and exploitation. Simulated Annealing, when hybridized with one of the Hill Climbing variants, demonstrated improved convergence properties. Additionally, the precision of at least 5 decimal places after 0 ensured accurate evaluation of the optimization results. Overall, this research provides valuable insights into the application of optimization algorithms in solving complex optimization problems and contributes to the understanding of their performance across different dimensions.

\section*{6 References}
\section*{References}
[1] Function Graphs images:\\
\href{https://www.sfu.ca/}{https://www.sfu.ca/} ssurjano/optimization.html\\
[2] Official informations about choosen functions:\\
\href{http://www-optima.amp.i.kyoto-u.ac.jp/member/student/hedar/}{http://www-optima.amp.i.kyoto-u.ac.jp/member/student/hedar/}\\
Hedar\_files/TestGO\_files/Page364.htm\\
[3] List formatation in LaTeX:\\
\href{https://tex.stackexchange.com/questions/2291/}{https://tex.stackexchange.com/questions/2291/}\\
how-do-i-change-the-enumerate-list-format-to-use-letters-instead-of-the-defaul\\
[4] How to write function in LaTeX:\\
\href{https://tex.stackexchange.com/questions/288124/}{https://tex.stackexchange.com/questions/288124/}\\
writing-functions-and-latex\\
[5] Trigonometric functions in LaTeX:\\
\href{https://www.geeksforgeeks.org/trigonometric-functions-in-latex/}{https://www.geeksforgeeks.org/trigonometric-functions-in-latex/}\\
[6] Mathematical Writting in LateX:\\
\href{https://ro.wikibooks.org/wiki/LaTeX_(carte)/Matematic%C4%83}{https://ro.wikibooks.org/wiki/LaTeX\_(carte)/Matematică}\\
[7] LaTeX formatting:\\
\href{https://tex.stackexchange.com/questions/294521/}{https://tex.stackexchange.com/questions/294521/}\\
getting-text-to-appear-on-the-same-line\\
[8] Adding mathematical symbols in LaTeX:\\
\href{https://oeis.org/wiki/List_of_LaTeX_mathematical_symbols}{https://oeis.org/wiki/List\_of\_LaTeX\_mathematical\_symbols}\\
[9] Adding images and shapes in LaTeX:\\
\href{https://stackoverflow.com/questions/3134187/}{https://stackoverflow.com/questions/3134187/}\\
how-to-add-a-jpg-image-in-latex\\
[10] Refferecing images in LaTeX:\\
\href{https://youtu.be/Ax9RCvjpI8E?si=udeNobkGVq_WbpOr}{https://youtu.be/Ax9RCvjpI8E?si=udeNobkGVq\_WbpOr}\\
[11] Creating and formatting tables in LaTeX:\\
\href{https://www.overleaf.com/learn/latex/Tables}{https://www.overleaf.com/learn/latex/Tables}\\
[12] Alghoritm descriptions and implementation details\\
\href{https://profs.info.uaic.ro/}{https://profs.info.uaic.ro/} eugennc/teaching/ga/\#Notions02


\end{document}