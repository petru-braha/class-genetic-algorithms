% heuristic and deterministic one

% De Jong 1, Schwefel's, Rastrigin's, Michalewicz's

% Michalewicz function 
n = 2;
m = 10;
s = 0;
for i = 1:n;
    s = s+sin(x(i))*(sin(i*x(i)^2/pi))^(2*m);
end
y = -s;

domain:
0 ≤ xi ≤ π

% Rastrigin function
n = 2; 
s = 0;
for j = 1:n
    s = s+(x(j)^2-10*cos(2*pi*x(j))); 
end
y = 10*n+s;

domain:
−5.12 ≤ xi ≤ 5.12

% Schwefel function
n = 2;
s = sum(-x.*sin(sqrt(abs(x))));
y = 418.9829*n+s;

domain:
−500 ≤ xi ≤ 500

% De Jong 1 function
f1(x)=sum(x(i)^2)

domain:
−5.12 ≤ xi ≤ 5.12


% n is the count of the array


% deterministic approach: take the domain and a step and compute for each value the function and then decide the best
for()

% heuristic approach: 
take the domain and choose randomly a canditate
compute neighbohor(canditate)
if(better)
    canditate = neighbohor
else
    stop


% hill climbing
t := 0
initialize best
repeat
    local := FALSE
    select a candidate solution (bitstring) vc at random
    evaluate vc
    repeat
        vn := Improve(Neghborhood(vc))
 	if eval(vn) is better than eval(vc)
	then vc := vn
	else local := TRUE
    until local
    t := t + 1
    if vc is better than best
    then best := vc
until t = MAX

Reprezentarea solutiilor:
    vectori de numere reale
    siruri binare: spatiul de cautare se va disctretiza pana la o anumita precizie 10-d.
    Un interval [a, b] va fi impartit in N = (b - a) * 10d subintervale egale.
    Pentru a putea reprezenta cele (b - a) * 10d valori, este nevoie de un numar n = parte_intreaga_superioara(log2(N)) de biti.
    Lungimea sirului de biti care reprezinta o solutie candidat va fi suma lungimilor reprezentarilor pentru fiecare parametru al functiei de optimizat.
    In momentul evaluarii solutiei (apelul functiei de optimizat) este necesara decodificarea fiecarui parametru reprezentat ca sir de biti in numar real, dupa formula:
    Xreal = a + decimal(xbiti) * (b - a) / (2n - 1)
    !!! Nu folositi reprezentari de forma vector de vectori de biti. Folositi un simplu vector de biti.
    Adica, reprezentati o solutie candidat ca un vector de biti, nu o matrice de biti.